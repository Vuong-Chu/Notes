
% Default to the notebook output style

    


% Inherit from the specified cell style.




    
\documentclass[11pt]{article}

    \usepackage{tcolorbox}
    
    \usepackage[T1]{fontenc}
    % Nicer default font (+ math font) than Computer Modern for most use cases
    \usepackage{mathpazo}

    % Basic figure setup, for now with no caption control since it's done
    % automatically by Pandoc (which extracts ![](path) syntax from Markdown).
    \usepackage{graphicx}
    \graphicspath{ {Figs/} }
    % We will generate all images so they have a width \maxwidth. This means
    % that they will get their normal width if they fit onto the page, but
    % are scaled down if they would overflow the margins.
    \makeatletter
    \def\maxwidth{\ifdim\Gin@nat@width>\linewidth\linewidth
    \else\Gin@nat@width\fi}
    \makeatother
    \let\Oldincludegraphics\includegraphics
    % Set max figure width to be 80% of text width, for now hardcoded.
    \renewcommand{\includegraphics}[1]{\Oldincludegraphics[width=.8\maxwidth]{#1}}
    % Ensure that by default, figures have no caption (until we provide a
    % proper Figure object with a Caption API and a way to capture that
    % in the conversion process - todo).
    \usepackage{caption}
    \DeclareCaptionLabelFormat{nolabel}{}
    \captionsetup{labelformat=nolabel}

    \usepackage{adjustbox} % Used to constrain images to a maximum size 
    \usepackage{xcolor} % Allow colors to be defined
    \usepackage{enumerate} % Needed for markdown enumerations to work
    \usepackage{geometry} % Used to adjust the document margins
    \usepackage{amsmath} % Equations
    \usepackage{amssymb} % Equations
    \usepackage{textcomp} % defines textquotesingle
    % Hack from http://tex.stackexchange.com/a/47451/13684:
    \AtBeginDocument{%
        \def\PYZsq{\textquotesingle}% Upright quotes in Pygmentized code
    }
    \usepackage{upquote} % Upright quotes for verbatim code
    \usepackage{eurosym} % defines \euro
    \usepackage[mathletters]{ucs} % Extended unicode (utf-8) support
    \usepackage[utf8x]{inputenc} % Allow utf-8 characters in the tex document
    \usepackage{fancyvrb} % verbatim replacement that allows latex
    \usepackage{grffile} % extends the file name processing of package graphics 
                         % to support a larger range 
    % The hyperref package gives us a pdf with properly built
    % internal navigation ('pdf bookmarks' for the table of contents,
    % internal cross-reference links, web links for URLs, etc.)
    \usepackage{hyperref}
    \usepackage{longtable} % longtable support required by pandoc >1.10
    \usepackage{booktabs}  % table support for pandoc > 1.12.2
    \usepackage[inline]{enumitem} % IRkernel/repr support (it uses the enumerate* environment)
    \usepackage[normalem]{ulem} % ulem is needed to support strikethroughs (\sout)
                                % normalem makes italics be italics, not underlines    
	\geometry{verbose,tmargin=1in,bmargin=1in,lmargin=1in,rmargin=1in}
	\usepackage{pstricks}
    
    
    % Colors for the hyperref package
    \definecolor{urlcolor}{rgb}{0,.145,.698}
    \definecolor{linkcolor}{rgb}{.71,0.21,0.01}
    \definecolor{citecolor}{rgb}{.12,.54,.11}

    % ANSI colors
    \definecolor{ansi-black}{HTML}{3E424D}
    \definecolor{ansi-black-intense}{HTML}{282C36}
    \definecolor{ansi-red}{HTML}{E75C58}
    \definecolor{ansi-red-intense}{HTML}{B22B31}
    \definecolor{ansi-green}{HTML}{00A250}
    \definecolor{ansi-green-intense}{HTML}{007427}
    \definecolor{ansi-yellow}{HTML}{DDB62B}
    \definecolor{ansi-yellow-intense}{HTML}{B27D12}
    \definecolor{ansi-blue}{HTML}{208FFB}
    \definecolor{ansi-blue-intense}{HTML}{0065CA}
    \definecolor{ansi-magenta}{HTML}{D160C4}
    \definecolor{ansi-magenta-intense}{HTML}{A03196}
    \definecolor{ansi-cyan}{HTML}{60C6C8}
    \definecolor{ansi-cyan-intense}{HTML}{258F8F}
    \definecolor{ansi-white}{HTML}{C5C1B4}
    \definecolor{ansi-white-intense}{HTML}{A1A6B2}

    % commands and environments needed by pandoc snippets
    % extracted from the output of `pandoc -s`
    \providecommand{\tightlist}{%
      \setlength{\itemsep}{0pt}\setlength{\parskip}{0pt}}
    \DefineVerbatimEnvironment{Highlighting}{Verbatim}{commandchars=\\\{\}}
    % Add ',fontsize=\small' for more characters per line
    \newenvironment{Shaded}{}{}
    \newcommand{\KeywordTok}[1]{\textcolor[rgb]{0.00,0.44,0.13}{\textbf{{#1}}}}
    \newcommand{\DataTypeTok}[1]{\textcolor[rgb]{0.56,0.13,0.00}{{#1}}}
    \newcommand{\DecValTok}[1]{\textcolor[rgb]{0.25,0.63,0.44}{{#1}}}
    \newcommand{\BaseNTok}[1]{\textcolor[rgb]{0.25,0.63,0.44}{{#1}}}
    \newcommand{\FloatTok}[1]{\textcolor[rgb]{0.25,0.63,0.44}{{#1}}}
    \newcommand{\CharTok}[1]{\textcolor[rgb]{0.25,0.44,0.63}{{#1}}}
    \newcommand{\StringTok}[1]{\textcolor[rgb]{0.25,0.44,0.63}{{#1}}}
    \newcommand{\CommentTok}[1]{\textcolor[rgb]{0.38,0.63,0.69}{\textit{{#1}}}}
    \newcommand{\OtherTok}[1]{\textcolor[rgb]{0.00,0.44,0.13}{{#1}}}
    \newcommand{\AlertTok}[1]{\textcolor[rgb]{1.00,0.00,0.00}{\textbf{{#1}}}}
    \newcommand{\FunctionTok}[1]{\textcolor[rgb]{0.02,0.16,0.49}{{#1}}}
    \newcommand{\RegionMarkerTok}[1]{{#1}}
    \newcommand{\ErrorTok}[1]{\textcolor[rgb]{1.00,0.00,0.00}{\textbf{{#1}}}}
    \newcommand{\NormalTok}[1]{{#1}}
    
    % Additional commands for more recent versions of Pandoc
    \newcommand{\ConstantTok}[1]{\textcolor[rgb]{0.53,0.00,0.00}{{#1}}}
    \newcommand{\SpecialCharTok}[1]{\textcolor[rgb]{0.25,0.44,0.63}{{#1}}}
    \newcommand{\VerbatimStringTok}[1]{\textcolor[rgb]{0.25,0.44,0.63}{{#1}}}
    \newcommand{\SpecialStringTok}[1]{\textcolor[rgb]{0.73,0.40,0.53}{{#1}}}
    \newcommand{\ImportTok}[1]{{#1}}
    \newcommand{\DocumentationTok}[1]{\textcolor[rgb]{0.73,0.13,0.13}{\textit{{#1}}}}
    \newcommand{\AnnotationTok}[1]{\textcolor[rgb]{0.38,0.63,0.69}{\textbf{\textit{{#1}}}}}
    \newcommand{\CommentVarTok}[1]{\textcolor[rgb]{0.38,0.63,0.69}{\textbf{\textit{{#1}}}}}
    \newcommand{\VariableTok}[1]{\textcolor[rgb]{0.10,0.09,0.49}{{#1}}}
    \newcommand{\ControlFlowTok}[1]{\textcolor[rgb]{0.00,0.44,0.13}{\textbf{{#1}}}}
    \newcommand{\OperatorTok}[1]{\textcolor[rgb]{0.40,0.40,0.40}{{#1}}}
    \newcommand{\BuiltInTok}[1]{{#1}}
    \newcommand{\ExtensionTok}[1]{{#1}}
    \newcommand{\PreprocessorTok}[1]{\textcolor[rgb]{0.74,0.48,0.00}{{#1}}}
    \newcommand{\AttributeTok}[1]{\textcolor[rgb]{0.49,0.56,0.16}{{#1}}}
    \newcommand{\InformationTok}[1]{\textcolor[rgb]{0.38,0.63,0.69}{\textbf{\textit{{#1}}}}}
    \newcommand{\WarningTok}[1]{\textcolor[rgb]{0.38,0.63,0.69}{\textbf{\textit{{#1}}}}}

    % Pygments definitions
    
\makeatletter
\def\PY@reset{\let\PY@it=\relax \let\PY@bf=\relax%
    \let\PY@ul=\relax \let\PY@tc=\relax%
    \let\PY@bc=\relax \let\PY@ff=\relax}
\def\PY@tok#1{\csname PY@tok@#1\endcsname}
\def\PY@toks#1+{\ifx\relax#1\empty\else%
    \PY@tok{#1}\expandafter\PY@toks\fi}
\def\PY@do#1{\PY@bc{\PY@tc{\PY@ul{%
    \PY@it{\PY@bf{\PY@ff{#1}}}}}}}
\def\PY#1#2{\PY@reset\PY@toks#1+\relax+\PY@do{#2}}

\expandafter\def\csname PY@tok@gd\endcsname{\def\PY@tc##1{\textcolor[rgb]{0.63,0.00,0.00}{##1}}}
\expandafter\def\csname PY@tok@gu\endcsname{\let\PY@bf=\textbf\def\PY@tc##1{\textcolor[rgb]{0.50,0.00,0.50}{##1}}}
\expandafter\def\csname PY@tok@gt\endcsname{\def\PY@tc##1{\textcolor[rgb]{0.00,0.27,0.87}{##1}}}
\expandafter\def\csname PY@tok@gs\endcsname{\let\PY@bf=\textbf}
\expandafter\def\csname PY@tok@gr\endcsname{\def\PY@tc##1{\textcolor[rgb]{1.00,0.00,0.00}{##1}}}
\expandafter\def\csname PY@tok@cm\endcsname{\let\PY@it=\textit\def\PY@tc##1{\textcolor[rgb]{0.25,0.50,0.50}{##1}}}
\expandafter\def\csname PY@tok@vg\endcsname{\def\PY@tc##1{\textcolor[rgb]{0.10,0.09,0.49}{##1}}}
\expandafter\def\csname PY@tok@vi\endcsname{\def\PY@tc##1{\textcolor[rgb]{0.10,0.09,0.49}{##1}}}
\expandafter\def\csname PY@tok@vm\endcsname{\def\PY@tc##1{\textcolor[rgb]{0.10,0.09,0.49}{##1}}}
\expandafter\def\csname PY@tok@mh\endcsname{\def\PY@tc##1{\textcolor[rgb]{0.40,0.40,0.40}{##1}}}
\expandafter\def\csname PY@tok@cs\endcsname{\let\PY@it=\textit\def\PY@tc##1{\textcolor[rgb]{0.25,0.50,0.50}{##1}}}
\expandafter\def\csname PY@tok@ge\endcsname{\let\PY@it=\textit}
\expandafter\def\csname PY@tok@vc\endcsname{\def\PY@tc##1{\textcolor[rgb]{0.10,0.09,0.49}{##1}}}
\expandafter\def\csname PY@tok@il\endcsname{\def\PY@tc##1{\textcolor[rgb]{0.40,0.40,0.40}{##1}}}
\expandafter\def\csname PY@tok@go\endcsname{\def\PY@tc##1{\textcolor[rgb]{0.53,0.53,0.53}{##1}}}
\expandafter\def\csname PY@tok@cp\endcsname{\def\PY@tc##1{\textcolor[rgb]{0.74,0.48,0.00}{##1}}}
\expandafter\def\csname PY@tok@gi\endcsname{\def\PY@tc##1{\textcolor[rgb]{0.00,0.63,0.00}{##1}}}
\expandafter\def\csname PY@tok@gh\endcsname{\let\PY@bf=\textbf\def\PY@tc##1{\textcolor[rgb]{0.00,0.00,0.50}{##1}}}
\expandafter\def\csname PY@tok@ni\endcsname{\let\PY@bf=\textbf\def\PY@tc##1{\textcolor[rgb]{0.60,0.60,0.60}{##1}}}
\expandafter\def\csname PY@tok@nl\endcsname{\def\PY@tc##1{\textcolor[rgb]{0.63,0.63,0.00}{##1}}}
\expandafter\def\csname PY@tok@nn\endcsname{\let\PY@bf=\textbf\def\PY@tc##1{\textcolor[rgb]{0.00,0.00,1.00}{##1}}}
\expandafter\def\csname PY@tok@no\endcsname{\def\PY@tc##1{\textcolor[rgb]{0.53,0.00,0.00}{##1}}}
\expandafter\def\csname PY@tok@na\endcsname{\def\PY@tc##1{\textcolor[rgb]{0.49,0.56,0.16}{##1}}}
\expandafter\def\csname PY@tok@nb\endcsname{\def\PY@tc##1{\textcolor[rgb]{0.00,0.50,0.00}{##1}}}
\expandafter\def\csname PY@tok@nc\endcsname{\let\PY@bf=\textbf\def\PY@tc##1{\textcolor[rgb]{0.00,0.00,1.00}{##1}}}
\expandafter\def\csname PY@tok@nd\endcsname{\def\PY@tc##1{\textcolor[rgb]{0.67,0.13,1.00}{##1}}}
\expandafter\def\csname PY@tok@ne\endcsname{\let\PY@bf=\textbf\def\PY@tc##1{\textcolor[rgb]{0.82,0.25,0.23}{##1}}}
\expandafter\def\csname PY@tok@nf\endcsname{\def\PY@tc##1{\textcolor[rgb]{0.00,0.00,1.00}{##1}}}
\expandafter\def\csname PY@tok@si\endcsname{\let\PY@bf=\textbf\def\PY@tc##1{\textcolor[rgb]{0.73,0.40,0.53}{##1}}}
\expandafter\def\csname PY@tok@s2\endcsname{\def\PY@tc##1{\textcolor[rgb]{0.73,0.13,0.13}{##1}}}
\expandafter\def\csname PY@tok@nt\endcsname{\let\PY@bf=\textbf\def\PY@tc##1{\textcolor[rgb]{0.00,0.50,0.00}{##1}}}
\expandafter\def\csname PY@tok@nv\endcsname{\def\PY@tc##1{\textcolor[rgb]{0.10,0.09,0.49}{##1}}}
\expandafter\def\csname PY@tok@s1\endcsname{\def\PY@tc##1{\textcolor[rgb]{0.73,0.13,0.13}{##1}}}
\expandafter\def\csname PY@tok@dl\endcsname{\def\PY@tc##1{\textcolor[rgb]{0.73,0.13,0.13}{##1}}}
\expandafter\def\csname PY@tok@ch\endcsname{\let\PY@it=\textit\def\PY@tc##1{\textcolor[rgb]{0.25,0.50,0.50}{##1}}}
\expandafter\def\csname PY@tok@m\endcsname{\def\PY@tc##1{\textcolor[rgb]{0.40,0.40,0.40}{##1}}}
\expandafter\def\csname PY@tok@gp\endcsname{\let\PY@bf=\textbf\def\PY@tc##1{\textcolor[rgb]{0.00,0.00,0.50}{##1}}}
\expandafter\def\csname PY@tok@sh\endcsname{\def\PY@tc##1{\textcolor[rgb]{0.73,0.13,0.13}{##1}}}
\expandafter\def\csname PY@tok@ow\endcsname{\let\PY@bf=\textbf\def\PY@tc##1{\textcolor[rgb]{0.67,0.13,1.00}{##1}}}
\expandafter\def\csname PY@tok@sx\endcsname{\def\PY@tc##1{\textcolor[rgb]{0.00,0.50,0.00}{##1}}}
\expandafter\def\csname PY@tok@bp\endcsname{\def\PY@tc##1{\textcolor[rgb]{0.00,0.50,0.00}{##1}}}
\expandafter\def\csname PY@tok@c1\endcsname{\let\PY@it=\textit\def\PY@tc##1{\textcolor[rgb]{0.25,0.50,0.50}{##1}}}
\expandafter\def\csname PY@tok@fm\endcsname{\def\PY@tc##1{\textcolor[rgb]{0.00,0.00,1.00}{##1}}}
\expandafter\def\csname PY@tok@o\endcsname{\def\PY@tc##1{\textcolor[rgb]{0.40,0.40,0.40}{##1}}}
\expandafter\def\csname PY@tok@kc\endcsname{\let\PY@bf=\textbf\def\PY@tc##1{\textcolor[rgb]{0.00,0.50,0.00}{##1}}}
\expandafter\def\csname PY@tok@c\endcsname{\let\PY@it=\textit\def\PY@tc##1{\textcolor[rgb]{0.25,0.50,0.50}{##1}}}
\expandafter\def\csname PY@tok@mf\endcsname{\def\PY@tc##1{\textcolor[rgb]{0.40,0.40,0.40}{##1}}}
\expandafter\def\csname PY@tok@err\endcsname{\def\PY@bc##1{\setlength{\fboxsep}{0pt}\fcolorbox[rgb]{1.00,0.00,0.00}{1,1,1}{\strut ##1}}}
\expandafter\def\csname PY@tok@mb\endcsname{\def\PY@tc##1{\textcolor[rgb]{0.40,0.40,0.40}{##1}}}
\expandafter\def\csname PY@tok@ss\endcsname{\def\PY@tc##1{\textcolor[rgb]{0.10,0.09,0.49}{##1}}}
\expandafter\def\csname PY@tok@sr\endcsname{\def\PY@tc##1{\textcolor[rgb]{0.73,0.40,0.53}{##1}}}
\expandafter\def\csname PY@tok@mo\endcsname{\def\PY@tc##1{\textcolor[rgb]{0.40,0.40,0.40}{##1}}}
\expandafter\def\csname PY@tok@kd\endcsname{\let\PY@bf=\textbf\def\PY@tc##1{\textcolor[rgb]{0.00,0.50,0.00}{##1}}}
\expandafter\def\csname PY@tok@mi\endcsname{\def\PY@tc##1{\textcolor[rgb]{0.40,0.40,0.40}{##1}}}
\expandafter\def\csname PY@tok@kn\endcsname{\let\PY@bf=\textbf\def\PY@tc##1{\textcolor[rgb]{0.00,0.50,0.00}{##1}}}
\expandafter\def\csname PY@tok@cpf\endcsname{\let\PY@it=\textit\def\PY@tc##1{\textcolor[rgb]{0.25,0.50,0.50}{##1}}}
\expandafter\def\csname PY@tok@kr\endcsname{\let\PY@bf=\textbf\def\PY@tc##1{\textcolor[rgb]{0.00,0.50,0.00}{##1}}}
\expandafter\def\csname PY@tok@s\endcsname{\def\PY@tc##1{\textcolor[rgb]{0.73,0.13,0.13}{##1}}}
\expandafter\def\csname PY@tok@kp\endcsname{\def\PY@tc##1{\textcolor[rgb]{0.00,0.50,0.00}{##1}}}
\expandafter\def\csname PY@tok@w\endcsname{\def\PY@tc##1{\textcolor[rgb]{0.73,0.73,0.73}{##1}}}
\expandafter\def\csname PY@tok@kt\endcsname{\def\PY@tc##1{\textcolor[rgb]{0.69,0.00,0.25}{##1}}}
\expandafter\def\csname PY@tok@sc\endcsname{\def\PY@tc##1{\textcolor[rgb]{0.73,0.13,0.13}{##1}}}
\expandafter\def\csname PY@tok@sb\endcsname{\def\PY@tc##1{\textcolor[rgb]{0.73,0.13,0.13}{##1}}}
\expandafter\def\csname PY@tok@sa\endcsname{\def\PY@tc##1{\textcolor[rgb]{0.73,0.13,0.13}{##1}}}
\expandafter\def\csname PY@tok@k\endcsname{\let\PY@bf=\textbf\def\PY@tc##1{\textcolor[rgb]{0.00,0.50,0.00}{##1}}}
\expandafter\def\csname PY@tok@se\endcsname{\let\PY@bf=\textbf\def\PY@tc##1{\textcolor[rgb]{0.73,0.40,0.13}{##1}}}
\expandafter\def\csname PY@tok@sd\endcsname{\let\PY@it=\textit\def\PY@tc##1{\textcolor[rgb]{0.73,0.13,0.13}{##1}}}

\def\PYZbs{\char`\\}
\def\PYZus{\char`\_}
\def\PYZob{\char`\{}
\def\PYZcb{\char`\}}
\def\PYZca{\char`\^}
\def\PYZam{\char`\&}
\def\PYZlt{\char`\<}
\def\PYZgt{\char`\>}
\def\PYZsh{\char`\#}
\def\PYZpc{\char`\%}
\def\PYZdl{\char`\$}
\def\PYZhy{\char`\-}
\def\PYZsq{\char`\'}
\def\PYZdq{\char`\"}
\def\PYZti{\char`\~}
% for compatibility with earlier versions
\def\PYZat{@}
\def\PYZlb{[}
\def\PYZrb{]}
\makeatother


% Exact colors from NB
\definecolor{incolor}{rgb}{0.0, 0.0, 0.5}
\definecolor{outcolor}{rgb}{0.545, 0.0, 0.0}
% Prevent overflowing lines due to hard-to-break entities
\sloppy 
% Setup hyperref package
\hypersetup{
      breaklinks=true,  % so long urls are correctly broken across lines
      colorlinks=true,
      urlcolor=urlcolor,
      linkcolor=linkcolor,
      citecolor=citecolor,
}
% Slightly bigger margins than the latex defaults
    
    % Define a nice break command that doesn't care if a line doesn't already
% exist.
\def\br{\hspace*{\fill} \\* }
% Math Jax compatability definitions
\def\gt{>}
\def\lt{<}
% Document parameters
\title{Quantitative Finance}


    \begin{document}
    \author{Vuong Chu}
    
    \maketitle
    
    This document is to summarize knowledge in Quant Finance, which is helpful to prepare for a quantitative finance job. 
    
\tableofcontents

\section{Probability 101}
\subsection{Probability}
Probability theory is derived from a small set of axioms – a minimal set of essential assumptions. An understanding these core concepts does provide additional insight.\\

\textit{Definition 1.1:} A random experiment is a physical situation whose outcome cannot be predicted until it is observed.\\

\textit{Definition 1.2:} A sample space, $\Omega$, is a set of possible outcomes of a random experiment.\\

\textit{Definition 1.3:} An event, $\omega$, is a subset of the sample space $\Omega$. An event may be any subsets of the sample space W (including the entire sample space), and the
set of all events is known as the event space.\\

\textit{Definition 1.4:} The set of all events in the sample space $\Omega$ is called the event space, and is denoted $\mathcal{F}$.\\

\emph{Example 1}: Tossing a coin is to create a random experiment. In there, we have $\Omega = \{H, T\}$ and $\mathcal{F} = \{\emptyset, \{H\}, \{T\}, \Omega \}$.\\

\textbf{Axiom 1.1} For any event $\omega \in \mathcal{F}, Pr(\omega) \leq 0.$\\

\textbf{Axiom 1.2} The probability of all events in the sample space $\Omega$ is unity, i.e. $Pr(\Omega)=1$.\\

\textit{Definition 1.5:} Let $A$ and $B$ be two sets, then the union is defined $A \cup B = \{x: x \in A or x \in B\}$.\\

\textit{Definition 1.6:} Let $A$ and $B$ be two sets, then the intersection is defined $A \cap B = \{x: x \in A and x \in B\}$.\\

\textit{Definition 1.7:} Let $A$ be a set, then the complement set, denoted $A^c = \{x:x \notin A\}.$\\

\textit{Definition 1.8:} Let $A$ and $B$ be two sets, then $A$ and $B$ disjoint if and only if $A \cup B = \emptyset$.\\

\textbf{Axiom 1.3} Let $\{A_i\}, i=1,2,...$ be a finite or countably infinite set\footnote{If $S$ is a finite set, then the number of elements in S is a unique number in $\mathbb{N} = \{1,2,...\}$. A $S$ set is countably infinite if there exists a bijective (one-to-one) function from the elements of $S$ to the natural numbers $\mathbb{N}$.} of disjoint events. $Pr(\bigcup^\infty_{i=1}A_i) = \sum^\infty_{i=1}Pr(A_i)$.\\

\textit{Definition 1.8:} A probability space is denoted using the tuple $(\Omega,\mathcal{F},Pr)$ where $\Omega$ is the sample space, $\mathcal{F}$ is the event space and $Pr$ is the probability set function which has domain $\omega \in \mathcal{F}$ and range $Pr(\omega) \in [0,1]$.\\


\subsection{Expectation}
\subsection{Generating functions}
\subsection{Branching process}
\subsection{Markov chains}

\section{Real Analysis}
\subsection{Sets and Functions}
\textit{Definition 2.1.1:} If $A$ and $B$ are nonempty sets, then the Cartesian product $AxB$ of $A$ and $B$ is the set of all ordered pairs $(a,b)$ with $a \in A$ and $b \in B$. That is,
\begin{equation*}
A x B := \{(a,b): a \in A, b \in B\}
\end{equation*}
\textit{Definition 2.1.2:} Let $A$ and $B$ be sets. Then, a function from $A$ to $B$ is a set $f$ of ordered pairs in $A x B$ such that for each $a \in A$ there exists a unique $b \in B$ with $(a,b) \in f$.\\
\textit{Definition 2.1.3:} If $E$ is a subset of $A$, then direct image of $E$ under $f$ is the subset $f(E)$ of $B$ given by
\begin{equation*}
f(E) := \{f(x) : x \in E\}
\end{equation*}
If $H$ is a subset of $B$, then the inverse image of $H$ under $f$ is the subset $f^{-1}(H)$ of $A$ given by
\begin{equation*}
f^{-1}(H) := \{x \in A : f(x) \in H\\\}
\end{equation*}
\textit{Definition 2.1.4:} Let $f : A \rightarrow B$ be a function from $A$ to $B$:\\
(a) The function $f$ is said to be injective (or to be one-one) if whenever $x_1 \neq x2$, then $f(x_1) \neq f(x_2)$. If $f$ is an injective function, we also say $f$ is an injection.\\
(b) The function $f$ is said to be surjective (or to map $A$ onto $B$) if $f(A)=B$; that is, if the range $R(f)=B$. If $f$ is a surjective function, we also say that $f$ is a surjection.\\
(c) If $f$ is both injective and surjective, then $f$ is said to be bijective. If $f$ is bijective, we also say that $f$ is a bijection.\\
\textit{Definition 2.1.5:} If $A \rightarrow B$ is a bijection of A onto B, then
\begin{equation*}
	g:=\{(b,a) \in B x A : (a,b) \in f\}
\end{equation*}
is a function on $B$ into $A$. This function $g$ is called the inverse function of $f$, and is denoted by $f^{-1}$.\\
\textit{Definition 2.1.6:} Cantor's Theorem: If $A$ is any set, then there is no surjection of $A$ onto the set $P(A)$ of all subsets of $A$.

Cantor's theorem implies the collection $P(N)$ of all subsets of natural numbers $N$ is uncountable.

\textit{Definition 2.1.7:} Let $n_0 \in N$ and let $P(n)$ be a statement for each natural nature $n \geq n_0$. Suppose that:
(1) The statement $P(n_0)$ is true.
(2) For all $k geq n_0$, the truth of $P(k)$ implies the truth of $P(k+1)$. Then, $P(n)$ is true for all $n \geq n_0$.\\

\subsection{Real Numbers}
\textit{Definition 2.2.1:} Algebraic Properties of $R$. On the set $R$ of real numbers, there are two binary operations, denoted by + and . and called addition and multiplication, respectively. These operation satisfy the following properties:
(A1) $a + b = b + a$ for all a, bin R (commutative property of addition).\\
(A2) $(a+b)+c=a+(b+c)$ for all a, b, c in R (associative property of addition).\\
(A3) There exists an element 0 in R such that 0 + a = a and a + 0 = a for all a in R (existence of a zero element).\\
(A4) For each a in R, there exists an element -a in R such that a + (-a) = 0 and (-a) + a = 0(existence of negative elements).\\
(M1) a.b = b.a for all a, b in R (commutative property of multiplication).\\
(M2) (a.b).c = a.(b.c) for a, b, c in R (associative property of multiplication)\\
(M3) There exists an element 1 in R distinct from 0 such that 1.a = a and a.1 = a for all a in R (existence of unit element).\\
(M4) For each a $\neq$ 0 in R, there exists an element 1/a in R such that a.(1/a) = 1 and (1/a).a = 1 (existence of reciprocals).\\
(D) a.(b+c) = (a.b) + (a.c) and (b+c).a = (b.a) + (c.a) for all a, b, c in R (distributive property of multiplication over addition).\\

\textit{Definition 2.2.2:} Elements of R that can be written in the form b/a where a, b $\in$ Z and a $\neq$ 0 are called rational numbers.One consequence is that elements of R that are not in Q became known as irrational numbers, meaning that they are not ratios of integers.\\

\textit{Definition 2.2.3:} Trichotomy Property of R: if a belongs to R, then exactly one of the following holds:
a \in P, a = 0, -a \in P.\\

\textit{Definition 2.2.4:} If a \in R is such that 0 \leq a < $\epsilon$ for every $\epsilon > 0$, then a=0.\\

\textit{Definition 2.2.5:} Bernoulli's Inequality: If x > -1, then
\begin{equation*}
	(1+x)^n \geq 1 + nx \text{ for all } n \in N
\end{equation*}

\textit{Definition 2.2.6:} Triangle Inequality: If a,b \in R, then $|a+b| \leq |a| + |b|$

\textit{Definition 2.2.7:} Let $a \in R$ and $\epsilon > 0$. Then, the $\epsilon-neighborhood$ of a is the set $V_\epsilon(a) := \{x \in R: |x-a|<\epsilon\}.$

\textit{Definition 2.2.8:} Let S be a nonempty subset of R\\
(a) The set S is said to be bounded above if there exists a number $u \in R$ such that $s \leq u$ for all $s \in S$. Each such number u is called an upper bound of S.\\
(b) The set S is said to be bounded below if there exists a number $w \in R$ such that $w \geq s$ for all $s \in S$. Each such number w is called the lower bound of S.\\
(c) A set is said to be bounded if it is both bounded above and bounded below. A set is said to be unbounded if it is not bounded.\\

\textit{Definition 2.2.9:} 



\subsection{Sequences and Series}
\textit{Definition 2.3.1:} A sequence of real numbers (or a sequence in R) is a function defined on the set N = {1,2,...} of natural numbers whose range is contained in the set R of real numbers.
Example: $X := (\frac{1}{2n}:n \in N)$\\

\textit{Definition 2.3.2:} A sequence $X = (x_n)$ in R is said to converge to $x \in R$, or x is said be a limit of $(x_n)$, if for every $\epsilon > 0$, there exists a natural number $K(\epsilon)$ such that all $n \geq K(\epsilon)$, the term $x_n$ satisfy $|x_n - x|<\epsilon$. If the sequence has no limit, we say that the sequence is divergent.\\

\textit{Definition 2.3.3:} A sequence $X = (x_n)$ of real numbers is said to be bounded if there exists a real number $M>0$ such that $|x_n| \leq M$ for all $n \in N$.

\textit{Definition 2.3.4:} A sequence $X = (x_n)$ of real numbers is said to be a Cauchy sequence if for every $\epsilon > 0$, there exists a natural number $K(\epsilon)$ such that for all numbers $n,m \geq K(\epsilon)$, the term $x_n, x_m$ satisfy $|x_n-x_m|<\epsilon0$.\\
This implies the terms of the sequence get closer to each other.\\

\textit{Definition 2.3.5:} If $X = (x_n)$ is a convergent sequence of real number, then $X$ is a Cauchy sequence.\\

\textit{Definition 2.3.6:} A Cauchy sequence of real number is bounded.\\

\textit{Definition 2.3.7:} A sequence of real number is convergent if and only if it is Cauchy sequence.\\

\subsection{Limits}
\subsection{Continuous Functions}
\subsection{Differentiation}
\subsection{Riemann Integral}
\subsection{Sequences of Functions}
\subsection{Infinite Series}
\textit{Definition 2.9.1:} If $X:= (x_n)$ is a sequence in R, then the infinite series (or simply the series) generated by X is the sequence $S:= (s_k)$ defined by:\\
\begin{align*}
	&s_1 := x_1\\
	&s_2 := s_1 + x_2 (= x_1 + x_2)\\
	&...\\
	&s_k := s_{k-1} + x_k (=x_1 + x_2 +...+x_k)\\
	&...
\end{align*}
We can use symbol such as: $\sum x_n$\\
\textit{Definition 2.9.2:} If the series $\sum x_n$ converges, then $lim(x_n) = 0$.
\subsection{Generalized Riemann Integral}
\subsection{Topology}


\section{Differential Equation}



\section*{Appendix}

\end{document}
